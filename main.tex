\documentclass[ucs,9pt]{beamer}

% Copyright 2004 by Till Tantau <tantau@users.sourceforge.net>.
%
% In principle, this file can be redistributed and/or modified under
% the terms of the GNU Public License, version 2.
%
% However, this file is supposed to be a template to be modified
% for your own needs. For this reason, if you use this file as a
% template and not specifically distribute it as part of a another
% package/program, I grant the extra permission to freely copy and
% modify this file as you see fit and even to delete this copyright
% notice.
%
% Modified by Tobias G. Pfeiffer <tobias.pfeiffer@math.fu-berlin.de>
% to show usage of some features specific to the FU Berlin template.

% remove this line and the "ucs" option to the documentclass when your editor is not utf8-capable
\usepackage[utf8x]{inputenc}    % to make utf-8 input possible
\usepackage[english]{babel}     % hyphenation etc., alternatively use 'german' as parameter


\include{fu-beamer-template}  % THIS is the line that includes the FU template!

% overwrite fu-beamer settings
\lstset{language=bash}
\lstset{mathescape=False}

\usepackage{arev,t1enc} % looks nicer than the standard sans-serif font
% if you experience problems, comment out the line above and change
% the documentclass option "9pt" to "10pt"

% image to be shown on the title page (without file extension, should be pdf or png)
% \titleimage{fu_500}


% \title[Short Paper Title] % (optional, use only with long paper titles)
\title
{My thought about RIOT build system}

% \subtitle
% {Include Only If Paper Has a Subtitle}

% \author[Author, Another] % (optional, use only with lots of authors)
\author[Gaëtan Harter] % (optional, use only with lots of authors)
{Gaëtan~Harter}
% - Give the names in the same order as the appear in the paper.

\institute[FU Berlin] % (optional, but mostly needed)
{Freie Universität Berlin}
% - Keep it simple, no one is interested in your street address.

\date[27/10/2017] % (optional, should be abbreviation of conference name)
{October 27, 2017}
% - Either use conference name or its abbreviation.
% - Not really informative to the audience, more for people (including
%   yourself) who are reading the slides online

\subject{RIOT build system}
% This is only inserted into the PDF information catalog. Can be left
% out.

% you can redefine the text shown in the footline. use a combination of
% \insertshortauthor, \insertshortinstitute, \insertshorttitle, \insertshortdate, ...
\renewcommand{\footlinetext}{\insertshortinstitute, \insertshorttitle, \insertshortdate}

% Delete this, if you do not want the table of contents to pop up at
% the beginning of each subsection:
\AtBeginSubsection[]
{
  \begin{frame}<beamer>{Outline}
    \tableofcontents[currentsection,currentsubsection]
  \end{frame}
}

\begin{document}

\begin{frame}[plain]
  \titlepage
\end{frame}

\begin{frame}{Outline}
  \tableofcontents
  % You might wish to add the option [pausesections]
\end{frame}


\section{Getting into RIOT build system}

\subsection{Building an app in one slide}

\begin{frame}{Building an app in one slide}
  It starts with the \texttt{APPLICATION} \texttt{Makefile}.

  \begin{itemize}
    \item \pause
      It sets \texttt{BOARD}, sets some \texttt{USEMODULE}, include \texttt{Makefile.include}

    \item \pause
      \texttt{Makefile.include} includes BOARDS/CPU Makefile.include

    \item \pause
      \texttt{Makefile.include} processes dependencies

    \item \pause
      \texttt{Makefile.include} exports variables listed in \texttt{vars.inc.mk}

    \item \pause
      \texttt{Makefile.include} executes \texttt{make -f application.inc.mk}

      \begin{itemize}
        \item \pause
          It builds the application library with \texttt{Makefile.base}
        \item \pause
          Then every real module is added to DIRS
        \item \pause
          And for every dir in DIRS execute \texttt{make -C \${module\_dir}}
      \end{itemize}

    \item \pause
      A module Makefile mainly only includes \texttt{Makefile.base}

    \item \pause
      Then all modules static libraries are linked together.

  \end{itemize}

\end{frame}


\subsection{Building a full distribution}

\begin{frame}{Building an equivalent of a linux distribution.}

  It is not about building only a kernel:

  \bigskip
  Its more about building a whole embedded distribution.

  \bigskip
  With a shell, a Webserver, your own applications.

\end{frame}


\begin{frame}{Building an equivalent of a linux distribution.}

  However, even if building a distribution:

  \begin{itemize}
    \item
      Still build it as an unique group of sources

    \item
      Module compilation is changed by the inclusion of other modules
      (CFLAGS, INCLUDES, \texttt{MODULE\_})

    \item
      Some modules set CFLAGS for all files (boards/cpu, ssp, embunit, \texttt{newlib\_gnu\_source})
  \end{itemize}

  \pause
  This for current design reasons, code size, compile time optimization,
  configuration management, build system limitations.

  \bigskip
  That is why, every app must build into its own 'bin/' directory.

\end{frame}


\begin{frame}{Building an equivalent of a linux distribution.}

  Some of the architecture could be changed, but it not the question here.

  \bigskip
  Solve build system problems and keep all current functionalities.

\end{frame}


\section{Points reviews}


\subsection{make without clean}

\begin{frame}{make without clean}

  \texttt{make BOARD=board\_name} should be enough all the time

  \begin{itemize}
    \item \pause
      Modifying source files, or headers rebuilds.
    \item
      Modifying \texttt{CFLAGS} rebuilds.
      \pause Doesn't it ?

      \begin{itemize}
        \item
          Not when \texttt{DEFAULT\_CHANNEL} sets \texttt{CFLAGS} after Makefile.include
        \item
          Also \texttt{\$(ASSMOBJ)} are not rebuilt now (missing dependency to header file).
      \end{itemize}

    \item \pause
      When deleting a C file, its object is kept in the static library

      \begin{itemize}
        \item
          An APPLICATION and a MODULE can even have the same name
      \end{itemize}

    \item \pause
      Changing any makefile does not trigger rebuild.
    \item
      Changing \texttt{INCLUDES} does not trigger rebuild

  \end{itemize}

\end{frame}


\subsection{Recursive Makefiles}

\begin{frame}{Recursive Makefiles}

  Recursive Makefiles means re-executing make when building a sub part.

  \bigskip \pause
  Pros:

  \begin{itemize}
    \item
      It uses independent build environment for every module, with only specifically 'exported' variables.
    \item
      Easy to maintain, no modules variables/targets polluting other modules
    \item
      Allows easy handling of \texttt{make clean all -j}
  \end{itemize}

\end{frame}

\begin{frame}{Recursive Makefiles}

  Recursive Makefiles means re-executing make when building a sub part.

  \bigskip
  Cons:

  \begin{itemize}
    \item
      It uses independent build environment for every module, with only specifically 'exported' variables.
    \item
      All targets are removed between environments
    \item
      Not possible to have build dependencies between different modules.
    \item
      Must try to build every module even if up to date.
    \item
      Hard to do module specific configuration from outside of its Makefile
    \item
      Make never has a global view of what is built.
  \end{itemize}

\end{frame}


\subsection{MODULE definition is spread in the repository}

\begin{frame}{MODULE definition is spread in the repository}

  A MODULE is defined by:

  \begin{itemize}
    \item
      its \texttt{Makefile}
    \item
      directory to build, set by an common \texttt{Makefile} (ex \texttt{sys/Makefile})

    \item
      global config in its \texttt{Makefile.include} or a common one (sets FLAGS)

    \item
      reasons for its inclusion in common \texttt{Makefile.dep} (!= deps)

  \end{itemize}

  \pause
  Also special 'pseudo' modules
  \begin{itemize}
    \item
      \texttt{pseudomodules.inc.mk} for ones that do not produce any library file.
    \item
      Behaviour defined by common \texttt{Makefile.include/Makefile.dep}

  \end{itemize}

\end{frame}



\subsection{Imperative and not declarative}

\begin{frame}[fragile]{Imperative and not declarative}
  The RIOT make build system looks more like bash scripts than makefiles.

  Parsing if things are in variables and modifying them.

  \begin{lstlisting}
  ifneq (,$(filter gcoap,$(USEMODULE)))
    USEPKG += nanocoap
    USEMODULE += gnrc_sock_udp
  endif
  \end{lstlisting}

  "Dependencies" are changed by the order of definition.

\end{frame}


\section{What evolutions I imagine}

\subsection{Adding things}

\begin{frame}{Adding things}

  As long as possible, stick to Make and keep retrocompatibility.

  \bigskip

  To work towards integration, take one aspect

  \begin{itemize}
    \item
      Documentation
    \item
      Review by the community
    \item
      Tools to help static tests the functionnality
    \item
      Integrate new feature but keep old behaviour too
    \item
      Migrate piece by piece
  \end{itemize}

\end{frame}


\subsection{Have as goal to remove recursive Makefiles}

\begin{frame}{Have as goal to remove recursive Makefiles}

  Replacing recursive makefiles, would mean replacing all

  \smallskip
  \texttt{make -C dir} by \texttt{include dir/Makefile}.

  \pause
  \begin{itemize}
    \item
      Needs adaptating targets and flags namespace

    \item
      Already shows bad makefile usage that could be solved.

    \item
      First, doing it for APPLICATION, would allow APP specific flags.
  \end{itemize}

  \pause
  I started playing with this.

\end{frame}


\subsection{What am I thinking when I say dependencies}

\begin{frame}{What am I thinking when I say dependencies}

  \begin{itemize}
    \item
      compile time dependencies

    \item
      compile time optional dependencies
    \item
      compile time optional dependencies that change the API

    \pause \smallskip

    \item
      Runtime dependencies ? (hardware ?, any others ?)

    \pause \smallskip

    \item
      Module defining 'interface'
    \item
      Module implementating an 'interface'

    \item
      Default implementation for 'interface'

    \smallskip

    \item
      ...
  \end{itemize}

\end{frame}


\subsection{One description per module}

\begin{frame}{One description per module}

  \begin{itemize}
    \item
      Have all information on a MODULE in its directory
    \item
      Have description for pseudomodules.
    \item
      Allow defining multiple modules in the same file.
  \end{itemize}

  \pause
  First real problem:
  \begin{itemize}
    \item
      direct impact on performance ?
    \item
      How to list them all ?
  \end{itemize}

\end{frame}


\subsection{Do as static as possible}

\begin{frame}{Do as static as possible}
  Remove the nember of variables setting by parsing USEMODULE.

  \bigskip
  Define all the dependencies/configuration per module
  \begin{itemize}
    \item
      \texttt{DEPENDS\_\$(MODULE) = }
    \item
      \texttt{MODULE\_DIR\_\$(MODULE) = }
  \end{itemize}

  And replace the previous code by generic resolution function.

  \bigskip
  Separate steps: definition, resolution one, resolution two

\end{frame}

\section{Thank you}

\begin{frame}{Thank you}

  Thank you for listening.

  \bigskip \pause
  Any questions ?
\end{frame}


% \begin{frame}{Using the build system in the context of the RIOT Appstore.}
%
%   My idea of an Appstore in few words:
%
%   \begin{itemize}
%
%     \item
%       For users, using a graphical interface
%       \begin{itemize}
%         \item
%           select board
%           \pause
%         \item
%           select App in an interface
%           \pause
%           (or more than one maybe ???)
%
%         \pause
%         \item
%           select optional modules
%         \pause
%         \item
%           do some configuration
%         \pause
%         \item
%           securely set his cloud credentials
%         \pause
%         \item
%           flash the given firmware
%       \end{itemize}
%
%   \end{itemize}
% \end{frame}
%
%
% \begin{frame}{Using the build system in the context of the RIOT Appstore.}
%
%   My idea of an Appstore in few words:
%
%   \begin{itemize}
%     \item
%       For developpers
%       \begin{itemize}
%         \item
%           \pause
%           Write an APPLICATION or a MODULE (wo should write 'main' ?)
%
%         \item
%           \pause
%           Write its dependencies
%
%         \item
%           \pause
%           Have support for optional dependencies
%
%         \item
%           \pause
%           Have support to add shell commands/init functions
%
%         \item
%           \pause
%           Declare possible configurations
%
%         \item
%           \pause
%           With its code outside of RIOT repository
%
%       \end{itemize}
%   \end{itemize}
%
% \end{frame}


\end{document}
